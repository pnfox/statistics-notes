% ------------------------------------------------------------------------------
% Chapter 2
% ------------------------------------------------------------------------------
\chapter{Discrete Probability Distributions}
\label{chapter2}

\par
In mathematics a variable is a symbol which represents a quantity, number, function, graph, or any mathematical object.
In computer science a variable is also a placeholder for quantities or (in higher level languages) objects. When a variables' values
are bound by randomness, we call these random variables. The random variable $X$ might represent the outcome of flipping a coin, while
the random variable $Y$ may be the price of milk next year.\newline \par

More formally, a \textbf{random variable} is a real-valued function whose domain is a sample space.\newline \par

In the case of our example variable $X$ the sample space is either heads or tails. For our variable $Y$ the sample space is any real number
(i.e. any number between $-\infty$ and $\infty$). When the sample space only constitues a finite set of values, we say that the random variable
is discrete. Otherwise it is called a continious random variable, even if the price of milk can only be between $0$ and $10$, the fact it lies randomly on
a continious interval means it is a continious random variable. To be more specific: \newline \par

A random variable $X$ is said to be \textbf{discrete} if it can take on only a finite number - or a countably infinite number - of possible values of x
The probability function of $X$, denoted $p(x)$, assigns probability to each value of x of $X$ so that the following conditions are satisfied.

\begin{enumerate}
	\item $P(X=x) = p(x) \geq 0$
	\item $\sum_{x}P(X=x)=1$ where the sum is over all possible values of x
\end{enumerate}\par

It is often useful in some cases to study random variables by looking at cumulative probabilities; that is, we
look at the probability a random variable $X$ takes a value less than or equal to some value $x$, i.e. $P(X\leq x)$.
This is described by the \textbf{cumulative distribution function}, denoted by $F(x)$.\newline \par

Hence the cumulative distribution function $F(x)$ for a randon variable $X$ is
\begin{align*}
	F(b)&=P(X\leq x) \\
	\textrm{if }&X\textrm{ is discrete then,} \\
	F(b)&=\sum_{x=-\infty}^{b}p(x)\\
\end{align*}
where $p(x)$ is the probability function.
